\documentclass[conference]{IEEEtran}
\IEEEoverridecommandlockouts
% The preceding line is only needed to identify funding in the first footnote. If that is unneeded, please comment it out.
\usepackage[spanish,es-nodecimaldot]{babel}
\usepackage{cite}
\usepackage{amsmath,amssymb,amsfonts}
\usepackage{algorithmic}
\usepackage{graphicx}
\usepackage{textcomp}
\usepackage{float}
\usepackage{xcolor}
\def\BibTeX{{\rm B\kern-.05em{\sc i\kern-.025em b}\kern-.08em
    T\kern-.1667em\lower.7ex\hbox{E}\kern-.125emX}}
\usepackage{hyperref}
\hypersetup{      % links in new PDF window
    colorlinks=true,        % color of external links
    urlcolor=black,
    citecolor=black,
    linkcolor=black}
\begin{document}

\title{Termómetro analógico con sensor de luz}

\author{\IEEEauthorblockN{Rubí Esmeralda Ramírez Milián}
\IEEEauthorblockA{\textit{Universidad de San Carlos de Guatemala} \\
\textit{Escuela de Ciencias Físicas y Matemáticas}\\
Guatemala, Guatemala \\
carné}
\and
\IEEEauthorblockN{Jorge Alejandro Ávalos}
\IEEEauthorblockA{\textit{Universidad de San Carlos de Guatemala} \\
\textit{Escuela de Ciencias Físicas y Matemáticas}\\
Guatemala, Guatemala \\
carné}
\and
\IEEEauthorblockN{Jorge Alejandro Rodriguez Aldana}
\IEEEauthorblockA{\textit{Universidad de San Carlos de Guatemala} \\
\textit{Escuela de Ciencias Físicas y Matemáticas}\\
Guatemala, Guatemala \\
201804766}

}

\maketitle

\begin{abstract}
    
\end{abstract}

\section{Objetivos}
Hola Rubi
\subsection{General}
    
\subsection{Específicos}
    \begin{enumerate}
        \item 
    \end{enumerate}

\section{Introducción}



\section{Marco teórico}



\section{Diseño Experimental}



\section{Resultados}



\section{Discusión de resultados}



\section{Conclusiones}



\begin{thebibliography}{00}
\bibitem{Wiki1} Diodo - Wikipedia, la enciclopedia libre. (2019, October 27). Retrieved from \url{https://es.wikipedia.org/wiki/Diodo}
\end{thebibliography}

\end{document}
